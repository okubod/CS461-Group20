\documentclass[10pt,draftclsnofoot,onecolumn]{IEEEtran}
\usepackage{cite}
\title{Mixed Reality Interface for Infrastructure Maintenance}
\author{Christopher Cooper, CS461-001, Fall 2018}
\date{October 2018}
\begin{document}

	\maketitle


	\newpage
	\section*{Problem in Maintenance Efficiency}

	New technologies in mixed reality, from here on refered to as MR,
	can provide new ways for those performing maintenance on infrastructure
	(roads, bridges, etc.). MR is a virtual space where real objects
	or people are integrated into a virtual environment and can co-exist and
	interact with the virtual objects in real time~\cite{MRdef}. Mixed reality
	provides users a way to take advantage of graphical computer processes and
	models without interfering with their vision of the environment and task at
	hand.

	In infrastructure maintenance, the scale of projects can be quite large and
	costly. This has led to wide use of Building Information Modelling in
	construction. Building Information Modeling is the process of the generation
	and management of information for construction projects~\cite{BIM}.
	The idea was introduced in the 70s under the name Building Description
	System but it has recently seen wide spread adoption in construction
	~\cite{BIM}. With the increased use of Building Information Modelling in
	construction, the main problem lies in the fact that existing structures
	can't be modeled in the same way~\cite{3DModelInf}. While the idea of a 3D
	model for a construction project is useful, the individuals performing the
	maintenance must constantly check with the model in order to make any use
	of it.

	\section*{A Solution in Mixed Reality}

	The proposed solution is to create an MR interface to allow maintanance
	professionals working on infrastructure to provide data and see a graphical
	display of the maintenance required, anchored to the real environment.
	It is intended to give the maintenance professionals a way to visually
	observe and intuitively interact with the data gathered on the repair area.
	This is in hopes of improving the overall speed and efficiency of the overall
	maintenance process, allowing more work to be done in less time and with
	minimal wasted resources.

	In addition, information on possible extensions to the system, or other
	problems in maintenance that MR could solve is desired. This would provide
	faster prototyping of future MR applications regarding infrastructural
	maintenance.

	The proposed software would comprise of two components: the interface for
	inputting data and manipulating digital elements, and the 3d representation
	of the data overlaying the environment. Both of these would be achieved
	using the Microsoft HoloLens or similar device that would allow interaction
	with the digital components using voice and hand gestures.

	\subsection*{The User Interface}

	The user interface would primarily be handled by gestures for controlling
	the orientation and location of digital models with minimal menus to
	control modes of the software such as inputing data or manipulation of the
	3D model. Manual data input would be handled with voice commands to easily
	convey to the program what type of data you are entering as well as the
	quantity without searching through complex menus.

	Data would also have the option to be input via upload from file. This is
	to handle any precompiled data that is in large quantities that would be
	undesirable to input manually.

	\subsection*{The 3D Model}

	The 3D model itself must anchor itself to a specified location that does
	not rely on constant visual contact with the area so that workers would not
	need to constantly reset the model orientation every time they reach for a
	tool or otherwise avert their gaze. In addition, the 3D model must have
	multiple viewing modes to allow workers to switch between seeing a solid
	representation of the desired end result and a more transparent view to not
	obstruct their line of sight in a convenient manner.

	In addition to the model of the actual structure, data should also be
	viewable visually in the form of a chart if such information should be
	necessary.

	\section*{Performance Metrics}

	At the end of the successful project there will be two deliverables, a
	functional prototype of the software and a report of the research regarding
	specific strategies to designing mixed reality solutions found in order
	to aid in future development of improved software.

	The prototype would include a functional interface utilizing both voice and
	gesture comands in an intuitive manner and 3D models generated from input
	data that anchors to the environment.
	The report would include descriptions of developed ideas that were and were
	not included in the delivered prototype with explanations of their
	respective viabilities in practice. This report would help understand the
	limitations of current mixed reality technology.

	\newpage
\bibliography{probstatebib}{}
\bibliographystyle{IEEEtran}
\end{document}
