\documentclass[onecolumn, draftclsnofoot,10pt, compsoc]{IEEEtran}
\usepackage{graphicx}
\usepackage{url}
\usepackage{setspace}
%\usepackage[style=numeric,sorting=nty]{biblatex}

%\addbibresource{mybib.bib}

%\title{Technology Review: Mixed Reality For Infrastructure Maintenance}
%\author{Team 20: Christopher C Cooper}
%\date{November 2018}

\usepackage{geometry}
\geometry{textheight=9.5in, textwidth=7in}

% 1. Fill in these details
\def \CapstoneTeamName{		xRLucid}
\def \CapstoneTeamNumber{		20}
\def \GroupMemberOne{			Christopher Cooper}
\def \GroupMemberTwo{			Austin Liang}
\def \GroupMemberThree{			David Okubo}
\def \GroupMemberFour{			Jonathan Chen}
\def \GroupMemberFive{			Mingyu Zhang}
\def \CapstoneProjectName{		Mixed Reality for Infrastructure Maintenance}
\def \CapstoneSponsorCompany{	OSU School of Civil and Construction Engineering}
\def \CapstoneSponsorPerson{		Yelda Turkan}

% 2. Uncomment the appropriate line below so that the document type works
\def \DocType{	%	Problem Statement
				%Requirements Document
				%Technology Review
				%Design Document
				Progress Report
				}
			
\newcommand{\NameSigPair}[1]{\par
\makebox[2.75in][r]{#1} \hfil 	\makebox[3.25in]{\makebox[2.25in]{\hrulefill} \hfill		\makebox[.75in]{\hrulefill}}
\par\vspace{-12pt} \textit{\tiny\noindent
\makebox[2.75in]{} \hfil		\makebox[3.25in]{\makebox[2.25in][r]{Signature} \hfill	\makebox[.75in][r]{Date}}}}
% 3. If the document is not to be signed, uncomment the RENEWcommand below
\renewcommand{\NameSigPair}[1]{#1}

%%%%%%%%%%%%%%%%%%%%%%%%%%%%%%%%%%%%%%%
\begin{document}
\begin{titlepage}
    \pagenumbering{gobble}
    \begin{singlespace}
    	\includegraphics[height=4cm]{coe_v_spot1}
        \hfill 
        % 4. If you have a logo, use this includegraphics command to put it on the coversheet.
        %\includegraphics[height=4cm]{CompanyLogo}   
        \par\vspace{.2in}
        \centering
        \scshape{
            \huge CS Capstone \DocType \par
            {\large November 26, 2018}\par
            \vspace{.5in}
            \textbf{\Huge\CapstoneProjectName}\par
            \vfill
            {\large Prepared for}\par
            \Huge \CapstoneSponsorCompany\par
            \vspace{5pt}
            {\Large\NameSigPair{\CapstoneSponsorPerson}\par}
            {\large Prepared by }\par
            Group\CapstoneTeamNumber\par
            % 5. comment out the line below this one if you do not wish to name your team
            \CapstoneTeamName\par 
            \vspace{5pt}
            {\Large
                \NameSigPair{\GroupMemberOne}\par
                \NameSigPair{\GroupMemberTwo}\par
                \NameSigPair{\GroupMemberThree}\par
                \NameSigPair{\GroupMemberFour}\par
                \NameSigPair{\GroupMemberFive}\par
            }
            \vspace{20pt}
        }

\begin{abstract}
This document provides a progress update from the last ten weeks.
The project, MR for Infrastructure Maintenance, will provide visualization of CAD models for the Civil and Construction Engineering industry.
The team has completed the purpose statement, requirements, technology reviews, and design for the project.
The team is also learning the development tools for the implementation of the project.
Though there are two problems for the project.
The first is the fact that the Autodesk Forge SDK does not allow for download of the model in a useful format for use in our application.
The second is that CAD models in general are not optimized for use in real-time rendering applications.
The first problem is solvable by requiring the model to be provided by the user in fbx format, while the second problem relies on the user to optimize the polygon count of the model themselves.
\end{abstract}
\end{singlespace}
\end{titlepage}
\newpage

\pagenumbering{arabic}
\tableofcontents
%\listoffigures
\listoftables
\clearpage

\section{Purpose}
This document is an update of the progress made on the MR for Infrastructure Maintenance project.
The MR for Infrastructure Maintenance project will develop software made to assist in project visualization in Civil and Construction Engineering for field and education.
Civil and Construction Engineering has recently started utilizing CAD models with associated meta-data in the form of a BIM.
The software is intended to aid in visualization of a BIM while in the field or education environments by utilizing mixed reality.\par

\section{Current Progress}
Through the past ten weeks, the purpose of the project was outlined, followed by the requirements and design.
The development team has also familiarized themselves to some extent with the chosen development tools, Unity Engine, Vuforia SDK, and Autodesk Forge SDK.
With the purpose, requirements, and design of the project wrapped up, the team will be waiting for confirmation from the client that implementation can begin.
The team will be familiarizing themselves with the software tools in the meantime, so as to allow coding to begin when confirmation has been given.\par

\section{Current Problems and Possible Solutions}
Currently the project faces two major problems.
The first is a limitation of the Autodesk Forge SDK in downloading models from their cloud service, while the second is the models typically created in CAD software not being optimized for real-time rendering.\par
The limitation of Autodesk not allowing for a download of their models in a useful format from the BIM 360 cloud service is easily remedied by loading models from the file system directly.
This puts the conversion of the model in the hands of the user, as they export the model from their CAD software in fbx format.
This format can then be loaded into our program at runtime.\par
The second problem, CAD models not being optimized for real-time rendering application, can't actually be solved in development.
This comes from the project loading an arbitrary model provided by the user and not using a predefined model.
The software will come with instruction and warning to limit polygon count of the model to be used.
This limitation will be kept in mind during development to leave as much resources for rendering as possible.\par

\section{Summaries of Weekly Activities}
% describe week by week activities, problems, solutions, and anything else that fits in this format
A summary of activities from each week is given in a week-by-week format. Problems that occurred during the week, as well as the solutions to those problems will also be presented.\par

\subsection{Week One}
Week one was a simple introduction to the course and the groups had not been formed at this point.
Work was individual assignment for resume building.\par

\subsection{Week Two}
The second week consisted of deciding and setting up groups for projects.
The team introduced themselves through email to each other at the end of the week.\par

\subsection{Week Three}
The third week began with the first meeting of the group.
Work on the problem statement.
First, an individual document was made by each member.
Later in the week, each individual document was combined to create a group document.
\par
As this was the first week working with the group, there were some difficulties figuring out a schedule that worked for every member.
The major problem that occurred this week was the team having trouble getting a meeting with the client.
This was solved by scheduling earlier with the client.\par

\subsection{Week Four}
In the fourth week, the group problem statement document was finished and turned in.
This was also the first week the group was able to meet with the client.
Details about the project in preparation for the requirements document.
This requirements document, explaining exactly what the project would entail, was started.\par
The major problem that occurred this week was the team having trouble getting a meeting with the client.
This was solved by scheduling earlier with the client.\par
While there were no problems this week, the team made an appointment to get more information about the project from the client.
This was needed for the requirements document.
Finally, the team outlined the basic requirements document.\par

\subsection{Week Five}
The fifth week started with a meeting with the TA.
This was to inform him of our project scope and initial problems.
The requirements document could not be progressed as much as the team wanted to as the client meeting was not until late in the week.\par
The communication with the client continued to be somewhat of a problem in that our schedules did not often line up.
The solution for this problem was making sure the team got in contact with the client as early as possible.\par

\subsection{Week Six}
The requirements document was completed in the beginning of the week.
The subjects for the tech review were decided in a meeting and the tech review documents were completed.\par
There were problems with starting on assignments a little later than preferable.
This was solved by better planning to get documents completed.\par

\subsection{Week Seven}
During week seven, the final drafts of the tech reviews were completed and discussed among the team members.
Some revisions to the requirements document were made.
With this week, the development tools for implementation were decided so the team began familiarizing themselves with these tools.\par
The client was unable to meet with the team this week and a meeting was set up for the following week.\par

\subsection{Week Eight}
The eighth week was a week for meetings.
The first meeting was with the client to discuss a software called Astralink.
This software was found by the team and appeared to cover the requirements of the client.
The client informed the team that they would contact the developers of the software and see if the software would work for them.
The client will contact the team with updates on the matter.
The second meeting was discuss acquisition of licenses for development tools that the team plans to use.\par
There were no problems that came up during this week.
This was mostly due to the week being uneventful.\par

\subsection{Week Nine}
The majority of the ninth week was work on the design document.
This included making final decisions on development tools to be used and general architecture of the design.
The design document progressed with few problems.\par

\subsection{Week Ten}
The tenth week was the last week for the term.
The week consisted of finishing the design document as well as preparing for the progress update, this document, to be completed.
With the lighter schedule of this week, further progress was made to familiarize the group with the chosen development tools for when implementation begins.\par
\newpage
\section{Retrospective}
This is a retrospective of the past ten weeks, providing the positives, negatives, and specific actions we will be implementing according to changes necessary for successful completion of the project.\par
\begin{table}[ht]
    \centering
    \begin{tabular}{p{1cm}|p{4.5cm}|p{4.5cm}|p{4.5cm}}
        \textbf{Week} & \textbf{Positives} & \textbf{Negatives} & \textbf{Actions} \\
        \hline \hline
        1 & Successful completion of individual work. & No problems occurred. & No actions from this week to be implemented going further. \\
        \hline
        2 & Assigned to group and introduced ourselves. & No problems occurred. & No actions from this week. \\
        \hline
        3 & Began work on project purpose statement. & Had difficulties getting a meeting with the client when we were in need of meeting with her. & Ensure organized and early communication with the client to avoid schedule conflicts occurring at bad times. \\
        \hline
        4 & Successful completion of group problem statement. Conceived rough idea of the project design. & Continued schedule conflicts with the client. & Same as week 3. \\
        \hline
        5 & Work progressed on requirements document. Successful meeting with the TA. & Continued schedule conflicts with the client, leading to a later than desired meeting. & Same as previous weeks. \\
        \hline
        6 & Requirements document was completed. Rough draft of tech reviews also completed. & Assignments started later than desired. & Schedules will be planned from the beginning going forward. \\
        \hline
        7 & Final draft of tech reviews completed. Discussed among team members. Development tools decided. & No problems occurred during this week. & No actions necessary from this week. \\
        \hline
        8 & Meeting to discuss Astralink software was held. Determined licenses required by development team. & No problems occurred this week. & No actions necessary from this week. \\
        \hline
        9 & Software architecture and development tools decided. & No problems occurred this week. & No actions necessary from this week. \\
        \hline
        10 & Completed design document of the project. Began work on progress update (this document). & No problems occurred during this week. & No actions necessary from this week. \\
    \end{tabular}
    \caption{Ten Week Retrospective}
    \label{tab:tab1}
\end{table}

\end{document}
