\documentclass[onecolumn, draftclsnofoot,10pt, compsoc]{IEEEtran}
\usepackage{graphicx}
\usepackage{url}
\usepackage{setspace}
%\usepackage[style=numeric,sorting=nty]{biblatex}

%\addbibresource{mybib.bib}

%\title{Technology Review: Mixed Reality For Infrastructure Maintenance}
%\author{Team 20: Christopher C Cooper}
%\date{November 2018}

\usepackage{geometry}
\geometry{textheight=9.5in, textwidth=7in}

% 1. Fill in these details
\def \CapstoneTeamName{		xRLucid}
\def \CapstoneTeamNumber{		20}
\def \GroupMemberOne{			Christopher Cooper}
\def \GroupMemberTwo{			Austin Liang}
\def \GroupMemberThree{			David Okubo}
\def \GroupMemberFour{			Jonathan Chen}
\def \GroupMemberFive{			Mingyu Zhang}
\def \CapstoneProjectName{		Mixed Reality for Infrastructure Maintenance}
\def \CapstoneSponsorCompany{	OSU School of Civil and Construction Engineering}
\def \CapstoneSponsorPerson{		Yelda Turkan}

% 2. Uncomment the appropriate line below so that the document type works
\def \DocType{	Problem Statement
				%Requirements Document
				%Technology Review
				%Design Document
				%Progress Report
				}

\newcommand{\NameSigPair}[1]{\par
\makebox[2.75in][r]{#1} \hfil 	\makebox[3.25in]{\makebox[2.25in]{\hrulefill} \hfill		\makebox[.75in]{\hrulefill}}
\par\vspace{-12pt} \textit{\tiny\noindent
\makebox[2.75in]{} \hfil		\makebox[3.25in]{\makebox[2.25in][r]{Signature} \hfill	\makebox[.75in][r]{Date}}}}
% 3. If the document is not to be signed, uncomment the RENEWcommand below
\renewcommand{\NameSigPair}[1]{#1}

%%%%%%%%%%%%%%%%%%%%%%%%%%%%%%%%%%%%%%%
\begin{document}
\begin{titlepage}
    \pagenumbering{gobble}
    \begin{singlespace}
    	\includegraphics[height=4cm]{coe_v_spot1}
        \hfill
        % 4. If you have a logo, use this includegraphics command to put it on the coversheet.
        %\includegraphics[height=4cm]{CompanyLogo}
        \par\vspace{.2in}
        \centering
        \scshape{
            \huge CS Capstone \DocType \par
            {\large November 26, 2018}\par
            \vspace{.5in}
            \textbf{\Huge\CapstoneProjectName}\par
            \vfill
            {\large Prepared for}\par
            \Huge \CapstoneSponsorCompany\par
            \vspace{5pt}
            {\Large\NameSigPair{\CapstoneSponsorPerson}\par}
            {\large Prepared by }\par
            Group\CapstoneTeamNumber\par
            % 5. comment out the line below this one if you do not wish to name your team
            \CapstoneTeamName\par
            \vspace{5pt}
            {\Large
                \NameSigPair{\GroupMemberOne}\par
                \NameSigPair{\GroupMemberTwo}\par
                \NameSigPair{\GroupMemberThree}\par
                \NameSigPair{\GroupMemberFour}\par
                \NameSigPair{\GroupMemberFive}\par
            }
            \vspace{20pt}
        }

\begin{abstract}
In this document there will be details discussing about Team 20's approach and understanding of Mixed Reality for Infrastructure Maintenance. There will also be details about how each member has analyzed the problem description and describe our understanding and implementation of Mixed Reality. As of now, Team 20's official definition for Mixed reality is that it utilizes technology used with virtual reality and augmented reality. It is much closer to augmented reality in its implementation as it involves the real world to build virtual objects in the real world space. However, mixed reality focuses on more interaction with the real world, thus making it ideal for construction maintenance. Lastly, we will focus on the technicalities of how the implementation will be structured and developed.
\end{abstract}
\end{singlespace}
\end{titlepage}
\newpage

\pagenumbering{arabic}
%\tableofcontents
%\listoffigures
%\listoftables
\clearpage

\section{Problem Description}
Building Information Model (BIM) files hold all the information of a physical structure from its conception phase to completion. These models are used to be shared and networked in order to support decision-making regarding the structure. The ease of access to building information models of a structure is lacking because of traditional practices of on hand and paper written 2D designs and blueprints. With technology quickly advancing and developing, 3D information models are now visually possible rather than using the traditional approach of 2D paper and written designs. With this development we would like to incorporate Mixed Reality into the solution, making virtual interactive construction maintenance both possible in future developments as well as being a much more efficient and accurate way to conduct construction maintenance. While there are some existing implementations of construction planning in virtual reality and alternate reality, they don't have the same interaction that mixed reality can provide.

%Building information models require a computer to access.
%This makes it difficult to use building information models in situations like on-site working on a structure or during times when a computer is not available like a meeting.
%The construction industry recently has started switching from traditional 2D objects to 3D objects for storing structure information.
%There is no real 'standard' for software that combines 3D objects and the structure information.


\section{Proposed Solution}
There is already technology used to scan real world objects and recreate 3D virtual models. Mixed reality could be implemented in such a way that a building’s blueprints could be recreated in a virtual space that corresponds with the physical building. This can let engineers examine and even manipulate certain parts of the architecture to determine what parts of the building need repair or possible improvements. The advantages of using this 3D blueprint over 2D blueprints is the different perspectives the user can view the building from, as well as more specific parts of the architecture that might need more inspection. \newline
Building inspection can be made easier if the user can access a virtual representation of what the building was originally planned to look like and function. They would essentially have a 3D blueprint applied to whatever part of the building they were looking at, making these inspections much faster and easier to understand. The software will be accessible through mobile devices, and would require a Building Information Model (BIM) to model the building in the software. This would then allow the user to view a mixed reality version of that building, allowing them to physically move their device to view different parts of the architecture, get specific information regarding the building's integrity, and find possible faults with the building that would require repairs or improvements. This would also help the engineer better visualize the plans with the building, as opposed to reading and interpreting a 2D design. Using a mobile device such as a phone or tablet would also ensure accessibility and portability of the software and its capabilities.

\section{Performance Metrics}
Upon project completion, a functional prototype will be deliverable to the client. The prototype will be a mobile application, specifically targeting the iOS/Android platforms on a tablet-sized device. The prototype will include the ability to load and save Building Information Model data, add/update specific project data, and display a 3D model of the project that anchors to a real location in the environment. \newline
This prototype may have some distance to our ideally final product, it at least should meet the performance criteria in user satisfaction, average response time, and error rates. \newline
For the user satisfaction, we aim 80\% satisfaction rate among testers during the acceptance test. The application we try to make is targeting the construction designer, and our team should consider their knowledge and working habits when designing the application’s interface. In the first time the new users use the application, they should learn quickly on how to work with it and be able to transform previous experience in using similar tools to it. As the acceptance test is performed, this is where we will gather a large number of testers from the Civil Engineering department to try out and find bugs or other errors in the code as a final product. During this test, the engineers will be asked about their satisfaction with the prototype as a proof of concept and we will aim for 80\% of responses to be positive. A more progressive development measurement that will be analyzed is the response time of actions in the program. The first measure is a comparison to current software for handling BIM, Autodesk Revit. \newline
For the response time, the time delay during each operation should not be noticeable by the user. There is two type of delay in the program, the first one is when loading the program or saving new changes. This stage would take a longer time to set or save a large amount of data, but it is one-time work and is not in our primary consideration. The major comparison of our work is to the desktop application, Autodesk Revit, and our loading time should less than twice the load time of Autodesk Revit. The second delay is during normal operation, like move a 3D object in a sense, an average response time should less than 0.1 seconds (100 ms). \newline
For the error rates, we plan to pass 100\% of the unit test and make sure the application doesn't crash during the acceptance test. Our primary goal in the area is no major code error and less functional error. For the code error, we will write a unit test to cover all part of the code and the program should pass it 100\%. And if it is finished early, we may make more test like mutation test and random test. For the functional error, we plan to test it several times when a prototype is finished and refine it if there is any bug or something doesn’t fit the requirement. And in acceptance test, there should be no bugs that crash the program. \newline
Through these goals, the prototype should end as a solid foundation for an expansion of the system in the future, utilizing new technologies as they are developed.


\end{document}
